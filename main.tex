\documentclass[submit,techrep]{ipsj}
\usepackage[dvips]{graphicx}
\usepackage{latexsym}
\def\Underline{\setbox0\hbox\bgroup\let\\\endUnderline}
\def\endUnderline{\vphantom{y}\egroup\smash{\underline{\box0}}\\}
\def\|{\verb|}
\setcounter{巻数}{53}%vol53=2012
\setcounter{号数}{10}
\setcounter{page}{1}
\begin{document}
\title{情報処理学会論文誌ジャーナル論文の準備方法\\(2012年10月12日版)}
\etitle{How to Prepare Your Paper for IPSJ Journal \\ (version 2012/10/12)}
\affiliate{IPSJ}{情報処理学会\\IPSJ, Chiyoda, Tokyo 101--0062, Japan}
\paffiliate{JU}{情報処理大学\\Johoshori Uniersity}
\author{情報 太郎}{Joho Taro}{IPSJ}[joho.taro@ipsj.or.jp]
\author{処理 花子}{Shori Hanako}{IPSJ}
\author{学会 次郎}{Gakkai Jiro}{IPSJ,JU}[gakkai.jiro@ipsj.or.jp]
\begin{abstract}
本稿は,情報処理学会論文誌ジャーナルに投稿する原稿を執筆する際,および論
文採択後に最終原稿を準備する際の注意点等をまとめたものである.大きく分け
ると,論文投稿の流れと,\LaTeX と専用のスタイルファイルを用いた場合の論
文フォーマットに関する指針,および論文の内容に関してするべきこと,するべ
きでないことをまとめたべからずチェックリストからなる.本稿自体も \LaTeX
と専用のスタイルファイルを用いて執筆されているため,論文執筆の際に参考に
\end{abstract}
\begin{jkeyword}
情報処理学会論文誌ジャーナル,\LaTeX,スタイルファイル,べからず集
\end{jkeyword}
\begin{eabstract}
This document is a guide to prepare a draft for submitting to IPSJ
Journal, and the final camera-ready manuscript of a paper to appear in
IPSJ Journal, using {\LaTeX} and special style files.  Since this
document itself is produced with the style files, it will help you to
refer its source file which is distributed with the style files.
\end{eabstract}
\begin{ekeyword}
IPSJ Journal, \LaTeX, style files, ``Dos and Dont's'' list
\end{ekeyword}
\maketitle
\section{はじめに}
\cite{latex}
\section{おわりに}
\begin{acknowledgment}
\end{acknowledgment}
\bibliography{main}
\bibliographystyle{ipsjunsrt}
\appendix
\end{document}
